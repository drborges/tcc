% Este documento utiliza latex classes disponibilizadas
% no seguinte endereço: http://www.inf.ufrgs.br/utug

\documentclass[cic,tc]{iiufrgs}

\usepackage[brazilian]{babel}
\usepackage{graphicx}
\usepackage[T1]{fontenc}
\usepackage{times}
\usepackage{float}
\usepackage{setspace}
\usepackage{listings}
\usepackage{color}
\usepackage{tabularx}
\usepackage[utf8]{inputenc}
\usepackage[alf,abnt-emphasize=bf]{abntex2cite}	% pacote para usar citações abnt


\title{Rivers - Stream Processing API for Golang}

\author{Borges}{Diego}

\advisor[Prof.~Dr.]{Pimenta}{Marcelo}

\date{dezembro}{2015}

\course{Curso de Ciência da Computação}

\location{Porto Alegre}{RS}

\graphicspath{{images/}}
\DeclareGraphicsExtensions{.png,.pdf}

\renewcommand{\nominata}{
UNIVERSIDADE FEDERAL DO RIO GRANDE DO SUL
\par Reitor: Prof.~Dr.~Carlos Alexandre Netto
\par Vice-Reitor: Prof.~Dr.~Rui Vicente Oppermann
\par Pró-Reitor de Graduação: Prof.~Dr.~Sérgio Roberto Kieling
\par Diretor do Instituto de Informática: Prof.~Dr.~ Luís da Cunha Lamb
\par Coordenador do Curso de Ciência da Computação: Prof.~Dr.~Raul Fernando Weber
\par Bibliotecária-Chefe do Instituto de Informática: Beatriz Regina Bastos Haro
}

\keyword{API}
\keyword{rivers}
\keyword{golang}
\keyword{stream}
\keyword{pipeline}
\keyword{concurrency}

%
% ---------------------------------------------------------
%


\begin{document}

\maketitle

\clearpage
\begin{flushright}
\mbox{}\vfill
{\sffamily\itshape
``Do not communicate by sharing memory; instead, share memory by communicating''\\}
--- \textsc{Rob Pike}
\end{flushright}

\chapter*{Agradecimentos}
\label{cha:thanks}
TODO ...
TODO ...
TODO ...

%
% ---------------------------------------------------------
%

\tableofcontents

\begin{listofabbrv}{UFRGS}
    \item[API] Application Programming Interface
    \item[CPU] Central Processing Unit
    \item[FIFO] First In First Out
    \item[TCP] Transmission Control Protocol
    \item[UDP] User Datagram Protocol
    \item[HTTP] Hypertext Transfer Protocol
    \item[NoSQL] Non SQL
    \item[MVP] Minimum Viable Product
    \item[TDD] Test-Driven Development
    \item[HTML] HyperText Markup Language
    \item[URL] Uniform Resource Locator
\end{listofabbrv}

\listoffigures

\listoftables

\begin{abstract}
Nos últimos anos o poder computacional evoluiu drasticamente e os sistemas atuais podem beneficiar-se de CPUs com múltiplos cores para realizar concorrentemente tarefas de maneira mais eficiente.

Golang tira proveito do poder computacional de hardwares modernos implementando um modelo simple porém poderoso de concorrência baseado em troca de mensagens através de channels e um formato mais barato de thread gerenciado pelo runtime conhecido como goroutines.

Rivers propoem um framework para processamento de fluxo de dados utilizando o modelo de concorrência da linguagem Go como fundação e padrões bem conhecidos como Producer-Consumer e Go pipeline pattern provendo uma API fluente e extensível para criação e composição de pipelines complexos de processamento de dados aplicando conceitos de programação funcional.
\end{abstract}

\begin{englishabstract}{Rivers - Stream Processing API for Golang}{API, rivers, golang, stream, pipeline, concurrency}
During the past few years hardware power has evolved drastically and today's systems can leverage multi-core CPUs in order to perform concurrent tasks more effectively.

Golang takes advantage of this hardware power by implementing a simple though extremely powerful concurrency model built on top of concepts such as message passing via channels and a more lightweight form of thread managed by the runtime known as goroutines.

Rivers proposes a framework for data stream processing built on top of Go's concurrency model along with well known patterns such as the Producer-Consumer pattern and the Go pipeline pattern in order to provide a fluent and extensible API for building and composing complex data processing pipelines through functional programming concepts.
\end{englishabstract}

\onehalfspacing

\chapter{Introdução}
\label{cha:introduction}

TODO ...

\section{Objetivos}
\label{sec:objectives}

TODO ...


\chapter{Fundamentos}
\label{cha:foundation}

Neste capítulo serão discutidos alguns conceitos básicos utilizados como fundação no desenvolvimento da solução assim como alguns detalhes técnicos relacionados a linguagem de programação Go necessários para o entendimento dos mecanismos implementados em Rivers e discutidos em capítulos seguintes.

\section{Streams}
\label{sec:streams}

Streams são definidos como uma sequência de dados que são produzidos assincronamente ao longo do tempo fluindo de sua fonte conhecida como upstream à um destino chamado de downstream \cite{article:tim:streams}. Frequentemente streams são comparados à coleções de dados como por exemplo arrays e listas. No entanto streams diferentemente de coleções não definem necessariamente um tamanho fixo, podendo produzir elementos indefinidamente ao longo do ciclo de vida de um programa. Apesar desta diferença, as APIs de streams são muitas vezes modeladas de maneira que se assemelham a APIs de manipulação de coleções compartilhando muitas das operações como \emph{filter}, \emph{map}, \emph{reduce} etc.

Streams são amplamente aplicados no âmbito computacional no entanto em muitos casos seus conceitos não são explicitamente aparentes. Um exemplo clássico são os \emph{Unix Pipes}, mecanismos utilizados para realizar comunicação entre processos \cite{book:tanenbaum:ipc} via canais de comunicação conhecidos como standard input e output ou simplesmente \emph{stdin} e \emph{stdout} respectivamente. Uma operação de pipe conecta o canal stdout de um programa ao stdin de outro permitindo com que dados sejam transmitidos de um lado ao outro passando por diferentes estágios de processamento específicos a cada programa. A figura \ref{fig:unix_pipes} mostra os comandos Unix \emph{find}, \emph{grep} e \emph{wc} sendo combinados via operações de \emph{pipe} formando um pipeline com três estágios de processamento.

As aplicações de streams são tantas que muitas linguagens de programação adotam completamente os seus conceitos e disponibilizam APIs robustas que permitem a criação de pipelines complexos de processamento de stream de dados, alguns exemplos são \cite{docs:nodejs:streams}, \cite{docs:haskell:streams} e \cite{docs:java8:streams}.

\begin{figure}[H]
  \includegraphics[width=0.55\textwidth]{unix-pipes}
  \centering
  \caption{Exemplo de stream pipeline em Unix.}
  \label{fig:unix_pipes}
\end{figure}

\section{Golang}
\label{sec:golang}

Go é uma linguagem de programação open source, estaticamente tipada \cite{paper:microsoft:static_typing} com suporte a garbage collector \cite{paper:sun:gc} criada pela Google com foco em simplicidade, produtividade e concorrência. O sistema de tipos da linguagem aborda de maneira inovadora alguns conceitos como por exemplo interfaces \cite{article:wikipedia:interfaces} que difere do conceito tradicional implementado por linguagens como Java. Outro aspecto interessante da linguagem é seu modelo de concorrência baseado em Goroutines \cite{docs:golang:goroutine} conceito similar ao de Coroutines \cite{article:wikipedia:coroutines} e troca de mensagens \cite{lecture:ucl:message_passing} através do uso de canais de comunicação. A seguir serão apresentados alguns dos conceitos da linguagem que foram essenciais na implementação de Rivers.

\subsection{Interfaces}
\label{subsec:interfaces}

Interface é o mecanismo através do qual sistemas podem ser modelados visando reuso e extensibilidade. Contratos abstratos são especificados descrevendo determinadas funcionalidades de um sistema independente de possíveis implementações, isso permite com que qualquer componente satisfazendo uma determinada interface possa ser utilizado como provedor da funcionalidade em questão. Em muitas linguagens de programação como Java, componentes precisam explicitamente declarar sua intenção de implementar uma interface e com isso sendo necessário definir as interfaces necessárias de um sistema como parte da modelagem da solução e consequentemente tornando a solução menos suscetível a futuras alterações de design. O código \ref{code:java:interfaces} mostra um exemplo de uma classe Java implementando uma interface específica.

\begin{figure}[H]
  \includegraphics[width=0.65\textwidth]{java_interfaces}
  \centering
  \caption{Exemplo de implementação de uma interface em Java.}
  \label{code:java:interfaces}
\end{figure}

Em Go, interfaces são satisfeitas implicitamente sem a necessidade de que componentes declararem explicitamente sua intenção de implementar uma interface eliminando assim hierarquias de dependências presentes em linguagens orientadas à objetos \cite{book:learn_oop}. Para que um componente A satisfaça a interface B, A deve simplesmente implementar todos os métodos declarados em B. Desta maneira decisões arquiteturais de sistema podem ser tomadas em momentos futuros a medida em que novos casos de uso são introduzidos e padrões de código detectados evoluindo assim a arquitetura gradativamente. O código a seguir mostra como o exemplo Java em \ref{code:java:interfaces} pode ser reescrito em Go. Note que o tipo Car não possui qualquer dependência com a interface Movable ao contrário da versão Java, porém pode ser utilizado como um tipo Movable por implementar o método MoveTo na linha 7.

\begin{figure}[H]
  \includegraphics[width=0.5\textwidth]{go_interfaces}
  \centering
  \caption{Exemplo de implementação de uma interface em Go.}
  \label{code:go:interfaces}
\end{figure}

Em \ref{sec:rivers:architecture} será discutido como Rivers faz o uso de interfaces para definir os building blocks do framework permitindo novas funcionalidades sejam introduzidas à arquitetura de maneira transparente e não intrusiva.

\begin{flushright}
\mbox{}\vfill
{\sffamily\itshape
``Note too that the elimination of the type hierarchy also eliminates a form of dependency hierarchy. Interface satisfaction allows the program to grow organically without predetermined contracts. And it is a linear form of growth; a change to an interface affects only the immediate clients of that interface; there is no subtree to update. The lack of implements declarations disturbs some people but it enables programs to grow naturally, gracefully, and safely.
''\\}
--- \textsc{Rob Pike}
\end{flushright}

\subsection{Goroutines}
\label{subsec:goroutines}

Goroutine é o mecanismo que a linguagem Go utiliza para executar código de maneira concorrente e potencialmente em paralelo de maneira similar a outros mecanismos como por exemplo \cite{article:wikipedia:coroutines} e \cite{article:wikipedia:threads} porém com algumas diferenças que faz com que seu papel no modelo de concorrência da linguagem seja crucial. Goroutines são basicamente funções que executam assincronamente e concorrentemente utilizando o comando go e são gerenciadas pelo runtime da linguagem. Ao contrário de threads que exigem pelo menos 1MB de memória inicial para inicialização, Goroutines são extremamente baratas sendo necessário apenas 2KB de memória para inicialização podendo ajustar este valor sob demanda alocando e desalocando espaço na Heap, permitindo com que centenas de milhares de Goroutines possam ser executadas concorrentemente. Além disso o custo de troca de contexto entre Goroutines é extremamente baixo comparado com threads uma vez que Goroutines são genrenciadas pelo runtime e ao contrário de threads não necessitam acessar recursos do Sistema Operacional para realizar a troca de contexto. \cite{blog:how_goroutines_work} faz uma excelente análise do funcionamento de Goroutines em comparação com o funcionamento de OS threads. A figura \ref{code:goroutine:example} mostra uma Goroutine sendo criada na linha 16 utilizando o comando go para executar assincronamente a função say com o parâmetro "world".

\begin{figure}[H]
  \includegraphics[width=0.75\textwidth]{goroutine_example}
  \centering
  \caption{Exemplo de uso de Goroutines.}
  \label{code:goroutine:example}
\end{figure}

\subsection{Channels}
\label{subsec:channels}

Channel é o mecanismo básico utilizado para realizar a comunicação entre Goroutines via troca de mensagem e a sincronização de suas execuções permitindo que dados sejam enviados de uma Goroutine à outra de maneira segura sem a necessidade de compartilhar memória e é a base para o modelo Produtor-Consumidor \cite{paper:david_kocher:producer_consumer} utilizado como fundação na implementação de Rivers.

Dados são escritos e lidos de um Channel de maneira síncrona sendo desnecessário o uso de outras primitivas de concorrência como por exemplo semáforos ou mutex \cite{paper:concurrency:schmidt}. Por padrão uma Goroutine ao escrever um dado em um Channel bloqueia sua execução até que outra Goroutine leia este dado do Channel liberando espaço para que outro dado seja escrito. No entanto um Channel pode ser criado com um Buffer permitindo com que vários dados possam ser escritos no Channel sem que sejam consumidos, bloqueando então apenas quando não houver mais espaço no Buffer. Channels podem ser de somente escrita, somente leitura ou ambos permitindo implementar alguns padrões de design interessantes como por exemplo Pipeline Pattern \cite{docs:golang:pipeline_pattern}. A figura a seguir mostra um exemplo de criação de um Channel com um Buffer de capacidade 2 na linha 2 e duas mensagens enviadas no canal nas linhas 3 e 4 e logo em seguida consumidas nas linhas 6 e 7. 

\begin{figure}[H]
  \includegraphics[width=0.55\textwidth]{channels_example}
  \centering
  \caption{Exemplo de uso de Channels.}
  \label{code:channels:example}
\end{figure}


\chapter{Trabalhos Relacionados}
\label{cha:related_work}

http://ahmetalpbalkan.github.io/go-linq/


\chapter{Rivers: Arquitetura}
\label{cha:rivers_architecture}

TODO - Brief Overview on the main architectural aspects:

\section{Building Blocks}
\label{sec:building_blocks}

TODO

Composition + Rivers Flow: From -> Apply -> Then

\section{Producers}
\label{sec:producers}

TODO ...

\section{Consumers}
\label{sec:consumers}

TODO ...

\section{Transformers}
\label{sec:transformers}

TODO ...

\section{Combiners}
\label{sec:combiners}

TODO ...

\section{Dispatchers}
\label{sec:dispatchers}

TODO ...

\section{Going Parallel}
\label{sec:going_parallel}

TODO ...

Concurrency is not parallelism

Talk about how Rivers achieve parallelism within a pipeline stage

Refer to http://blog.golang.org/concurrency-is-not-parallelism, use the gopher diagrams to make the idea clear

Benchmarks?

\section{The Error Recovering Mechanism}
\label{sec:recovering_mechanism}

TODO

Error scenarios + deadline timeouts

Golang Panic + Recover

Overriding the recovering function

The Debug Mode + Stacktrace

\section{Extending Rivers}
\label{sec:extending_rivers}

TODO

\subsection{Writing Custom Producers}
\label{sec:custom_producers}

TODO

\subsection{Writing Custom Consumers}
\label{sec:custom_consumers}

TODO

\subsection{Writing Custom Combiners}
\label{sec:custom_combiners}

TODO

\subsection{Writing Custom Dispatchers}
\label{sec:custom_dispatchers}

TODO


\chapter{Processo de Desenvolvimento}
\label{cha:rivers_implementation}

Rivers surgiu da necessidade de se implementar soluções simples porém eficientes para o processamento de grandes volumes de dados. Na empresa Bearch Inc esse processo é recorrente e pela falta de uma solução nativa no contexto da linguagem de programação Go utilizada amplamente nos projetos internos da empresa, desenvolvedores acabavam por implementar lógicas de processamento redundantes que se repetiam ao longo do desenvolvimento de vários projetos. Devido aos custos deste retrabalho e aos padrões encontrados no processo de desenvolvimento de vários projetos, foi proposto uma solução para ajudar a reduzir a quantidade de código necessário para se implementar essas rotinas de processamento assim como possibilitar o reuso de lógicas existentes de soluções anteriores.

Este capítulo discute brevemente o processo de desenvolvimento empregado assim como as práticas utilizadas para guiar a evolução da solução minimizando a quantidade de retrabalho necessário ao longo das iterações.

\section{Coleta de Requisitos e Roadmap}
\label{sec:requirements}

Requisitos foram levantados afim de se ter um conjunto mínimo de funcionalidades que pudesse formar o MVP inicial para que se iniciasse o desenvolvimento. A tabela \ref{mvp:requirements} lista as features consideranas no Roadmap de Rivers, algumas delas selecionadas para compor o MVP e outras implementadas somente na versão seguinte. As features marcadas para compor o MVP foram selecionadas com o intuito de se ter o mínimo de funcionalidades disponíveis para se implementar um pipeline de processamento de streams que possa ser utilizado em diferentes contextos já identificados em projetos anteriores da empresa, como por exemplo no processamento de entidades da base de dados assim como no processamento de resultados de chamadas de APIs de outros sistemas.

\begin{table}[h!]
    \centering
    \begin{tabular}{||c c c||} 
        \hline
        Backlog & MVP & V2.0 \\ [0.5ex] 
        \hline\hline
        Contexto Global & X & \\ 
        \hline
        Producer Interface & X & \\ 
        \hline
        Producers Especializados (List, Socket, File...) & & X \\ 
        \hline
        Transformer Interface & X & \\ 
        \hline
        Transformers Especializados (Map, Each, Filter...) & & X \\ 
        \hline
        Consumer Interface & X & \\
        \hline
        Consumers Especializados (Count, Reduce...) & & X \\
        \hline
        Dispatchers & & X \\
        \hline
        Combiners & & X \\
        \hline
        Panic Recovering & X & \\
        \hline
        Failure Retries & & X \\
        \hline
        Suporte a Paralelização & & X \\
        \hline
        Pipelines Distribuídos & & X \\ [1ex]
        \hline
    \end{tabular}
    \caption{Rivers Roadmap: MVP vs. V2.0}
    \label{mvp:requirements}
\end{table}

Afim de atender as necessidades dos projetos utilizados como casos de uso, foi decidido então que uma solução útil e viável teria que prover pelo menos uma implementação genérica de cada um dos estágios que compõem um pipeline. Um Producer deveria permitir com que diferentes fontes de dados sejam utilizadas no pipeline como geradores de dados como por exemplo base de dados, APIs Restful e arquivos. Uma implementação de Transformer deve permitir que uma lógica específica de processamento possa ser aplicada aos dados sendo transmitidos pelo stream e seu resultado passado ao próximo estágio. Um Consumer por sua vez deve permitir que dados possam ser coletados de maneira síncrona ao final do pipeline assim como possíveis erros de execução. Por fim, a implementação inicial deveria prover um mecanismo simples que pudesse detectar falhas em tempo de execução notificando-as aos estágios do pipeline para que os mesmos possam suspender sua execução.  Implementações mais especializadas de cada um destes estágios seriam o foco em versões futuras da API como por exemplo producers especializados em leitura de arquivos, drivers de base de dados específicas, transformers especializados em operações como Map, Reduce, Filter, etc, mecanismos para se executar o pipeline de maneira distribuída em um cluster de máquinas.

O plano de desenvolvimento foi mantido e disponibilizado como GitHub \ref{github} issues apresentado através de um dashboard Kanban \ref{kanban} disponível em Waffle.io \ref{waffle} para facilitar a visualização do progresso mantendo a informação e comunicação centralizada, como mostrado na figura \ref{fig:waffle}. Ao longo do desenvolvimento, cada feature implementada era prontamente testada em casos de uso reais por outros desenvolvedores e feedbacks coletados afim de aprimorar a solução de acordo com as necessidades reais do time.

\begin{figure}[H]
  \includegraphics[width=1\textwidth]{waffle}
  \centering
  \caption{Roadmap visualizado em um dashboard Waffle.}
  \label{fig:waffle}
\end{figure}

\section{Tests \& Benchmarks}
\label{sec:tests_n_benchmarks}

Foi utilizada a técnica Test Driven Development \cite{book:tdd:kent_beck} para toda nova funcionalidade implementada. Escrevendo-se os casos de teste antes mesmo de se ter a funcionalidade ajudou a guiar o design da API gradativamente, uma vez que toda complexidade envolvida na implementação da nova funcionalidade era colocado inicialmente de lado, permitindo o desenvolvedor focar no design da API se colocando como usuário da mesma em um primeiro momento. Através do uso desta técnica, foi possível alcançar uma cobertura de testes razoável criando um conjunto de testes de regressão muito útil na detecção de quebra de funcionalidades já existentes devido a alterações de certas áreas do codebase. Visando tirar o máximo de proveito da técnica, foi utilizado a ferramenta fsnotify \cite{tools:fsnotify} para execução dos testes de regressão sempre que um arquivo no codebase é alterado, disponibilizando um relatório dos resultados de cada teste executado com isso um feedback instantâneo com relação as últimas alterações no codebase. A figura \ref{fig:tdd-cycle} representa o ciclo de desenvolvimento seguido aplicando a técnica TDD.

\begin{figure}[H]
  \includegraphics[width=0.4\textwidth]{tdd-cycle}
  \centering
  \caption{Ciclo de desenvolvimento aplicando a técnica TDD.}
  \label{fig:tdd-cycle}
\end{figure}

Alguns benchmarks foram implementados para medir a eficiência de um pipeline segundo a solução proposta em comparação com soluções utilizada previamente em outros projetos. A tabela a seguir mostra os resultados medidos entre duas versões de um Job que processa 1000 entidades datastore por vez aonde o processamento leva em média 1 seg por entidade. Em uma versão foi implementado um pipeline Rivers com paralelização ativada afim de tirar o máximo de proveito do hardware utilizado. A outra implementação segue uma solução sequencial utilizada anteriormente em outros projetos.

\begin{table}[h!]
    \centering
    \begin{tabular}{||c c c||} 
        \hline
        Jobs & Número de Goroutines & Tempo Médio de Execução (seg) \\ [0.5ex] 
        \hline\hline
        Pipeline Rivers & 1000 & 1.16 \\ 
        \hline
        Solução Sequencial & 1 & >1000 \\ [1ex]
        \hline
    \end{tabular}
    \caption{Comparação entre Jobs sequencial e Rivers paralelizado}
    \label{benchmarks:sequential_vs_parallel}
\end{table}

\chapter{Rivers: Applicações Reais}
\label{cha:applications}

\section{Appx - Appengine eXtensions}
\label{sec:appx}

\emph{Appx} \cite{docs:drborges:appx} é um conjunto de extensões da plataforma \emph{Appengine} da Google \cite{docs:google:appengine} para o runtime da linguagem Go que oferece uma solução similar à Object Relational Mapping \cite{article:scott:orm} para modelagem da camada de domínio de aplicações que utilizam a solução de \emph{NoSQL} \cite{article:fowler:nosql} Datastore \cite{docs:google:datastore} da plataforma. Rivers foi utilizado para dar suporte à \emph{Continuous Querying Over Data Stream} \cite{paper:badu:continuous_querying}, permitindo que grandes conjuntos de dados possam ser continuamente buscados da base de dados em batches via paginação e processados assincronamente como um stream de entidades Datastore em poucas linhas de código.

Processar um grande conjunto de dados aplicando filtros, mapeamentos e realizando a paginação dos resultados é relativamente complexo quando se utilizando apenas as APIs nativas do Datastore. Desenvolvedores precisam explicitamente implementar a paginação contínua dos dados através da manipulação de cursores de busca, realizar tratamento de possíveis erros de execução além de implementar a lógica de processamento dos dados, sendo necessário várias linhas de código para se obter o resultado final levando a uma solução difícil de manter e muita duplicação de código quando esse processamento deve ser aplicado à diferentes entidades da base de dados. A figura \ref{code:datastore:pagination} mostra a implementação de um agendador de reuniões que envia email, sms e atualiza o calendário de cada empregado de uma empresa cujo título seja \emph{Manager} ou \emph{Director} com as informações da reunião. Devido a uma limitação do Datastore, é possível processar no máximo 1000 entidades por vez, sendo necessário realizar a paginação contínua dos dados até que todos eles sejam processados. Grande parte desta implementação não possui relação direta com a lógica necessária para o agendamento da reunião mas sim com a paginação dos dados.

\begin{figure}[H]
  \includegraphics[width=1\textwidth]{datastore_pagination}
  \centering
  \caption{Exemplo de uso da API nativa do Datastore.}
  \label{code:datastore:pagination}
\end{figure}

Appx utiliza Rivers para reduzir grande parte desta complexidade sendo necessárias poucas linhas de código para se obter o mesmo resultado do exemplo anterior. A figura \ref{code:appx:stream} mostra como podemos reescrever este mesmo caso de uso utilizando a API de streamming de Appx. Devido à arquitetura extensível de Rivers sua integração com Appx é relativamente simples, sendo necessário que Appx apenas implemente a interface \emph{stream.Producer} de Rivers que encapsula a execução da query, tratamento de erros e a paginação contínua dos resultados emitindo cada entidade datastore que satisfaz a query em um \emph{stream.Readable} ao qual operações como \emph{Filter}, \emph{Each}, \emph{Map} e \emph{Reduce} podem ser aplicadas formando um pipeline de processamento de entidades \emph{Datastore}. Desta maneira o resultado final não é somente mais legível mas como também fácil de manter e alterações na lógica de processamento é tão simples como adicionar ou remover filters, mappers, etc. 

\begin{figure}[H]
  \includegraphics[width=1\textwidth]{appx_stream}
  \centering
  \caption{Exemplo de uso da API de streamming do framework Appx.}
  \label{code:appx:stream}
\end{figure}

\section{Web Scrapping}
\label{sec:web_scrapping}

Web Scraping \cite{article:webharvy:scrapping} é uma solução simples, barata e eficaz para coleta e extração de dados na web e frequentemente são aplicados em conjunto com \emph{Web Crawlers} \cite{article:howstuffworks:web_crawler}, sistemas responsáveis por automatizar a busca e descoberta da informação, basicamente acessando documentos web através de \emph{URLs}, coletando e visitando cada URL no documento retornado aplicando este processo recursivamente até que uma determinada condição de parada seja satisfeita, afim de extrair e agregar determinados tipos de informação presentes em cada documento.

Este processo de crawling por envolver altos níveis de tráfego de rede pode vir a ser muito custoso e a possibilidade de paralelizar partes do processo pode ajudar a reduzir este custo. Rivers pode ser utilizado na implementação de pipelines com foco em Web Crawling e Scrapping, aonde cada crawler implementa a interface \emph{stream.Producer} com uma lógica específica responsável por "caminhar" a web seguindo determinadas regras específicas do crawler passando cada documento encontrado ao próximo estágio do pipeline que implementa a interface \emph{stream.Transformer} responsável por extrair as informações necessárias do documento implementando a função de scrapping. A imagen \ref{code:crawler} mostra um exemplo de Crawler utilizado como \emph{stream.Producer} em um pipeline Rivers responsável por extrair URLs de documentos HTML com a listagem de gastos públicos das cidades do Rio Grande do Sul disponíveis no \cite{portal_transparencia}, passando cada URL ao estágio seguinte do pipeline que implementa um scrapper \ref{code:scrapper} que dado uma URL o documento correspondente é recuperado via HTTP e as informações de gastos extraídas e coletadas em um dicionário chave-valor aonde a chave representa a área correspondente ao gasto como por exemplo educação ou saúde, e seu valor o total investido nesta área. A figura \ref{code:scrapping_example} mostra o uso do Crawler e Scrapper em um pipeline Rivers com paralelização ativa, podendo processar até 510 URLs em paralelo que é o número total de cidades a serem processadas e representa a capacidade máxima de produção do pipeline especificado pelo Crawler na linha 25.

\begin{figure}[H]
  \includegraphics[width=1\textwidth]{crawler}
  \centering
  \caption{Exemplo de um Crawler que implementa a interface stream.Producer.}
  \label{code:crawler}
\end{figure}

\begin{figure}[H]
  \includegraphics[width=1\textwidth]{scrapper}
  \centering
  \caption{Exemplo de um Scrapper que implementa a interface stream.Transformer.}
  \label{code:scrapper}
\end{figure}

\begin{figure}[H]
  \includegraphics[width=1\textwidth]{scrapping_example}
  \centering
  \caption{Exemplo de uso do Crawler e Scrapper em um pipeline Rivers.}
  \label{code:scrapping_example}
\end{figure}

\chapter{Conclusão}
\label{cha:conclusion}

Rivers propões uma API simples, extensível e intuitiva para processamento de streams de dados para a linguagem Go, utilizando de construções e conceitos de programação funcional familiares a maioria dos desenvolvedores abstraindo as complexidades e detalhes de implementação relacionados ao processamento concorrente e paralelo dos dados, de maneira que o desenvolvedor passa se concentrar apenas na lógica de negócio intrínseca ao pipeline.

O modelo de concorrência da linguagem Go provou-se não somente muito conveniente mas também extremamente eficiente e foi crucial para o sucesso da implementação da solução uma vez que muito da complexidade dos mecanismos de detecção de falhas, de comunicação entre os componentes do sistema via channels e gerência de recursos alocados durante a execução de grandes quantidades de fluxos de processamento concorrentes não afetaram de maneira considerável a complexidade final da solução.

Por se tratar de um experimento e guiado inicialmente pelas necessidades e casos de uso da emprese Bearch Inc, a API sofreu várias alterações ao longo do desenvolvimento afim de adaptá-la a novos casos de uso, porém visando simplicidade algumas operações encontradas em APIs similares em outras plataformas não foram implementadas como por exemplo determinados tipos de operações de filtros que no entanto podem ser implementadas através da composição de outras operações disponíveis na API.

\section{Análise dos Resultados}
\label{sec:results}

Ao longo do desenvolvimento muitas informações foram coletadas desde benchmarks até feedbacks de desenvolvedores utilizando o framework em outros projetos da empresa afim de avaliar a solução em termos de performance, simplicidade e flexibilidade. Os resultados dos benchmarks foram muito satisfatórios e mostraram um ganho considerável em performance ao se paralelizar estágios do pipeline além da execução concorrente de cada estágio.

O design funcional da API agradou desenvolvedores pela simplicidade da solução e o fato de se poder estender a API com novos componentes implementando as interfaces definidas pelo framework possibilitou o uso de Rivers em contextos bem variados. Algumas das decisões técnicas quanto ao design da API foram guiados por feedbacks de usuários que alongo de várias conversas e experimentos ajudaram a moldar a API final desde a nomenclatura das operações até mesmo com relação a melhor solução para se paralelizar pipelines sem expor qualquer tipo de complexidade ao usuário do framework. Um ponto negativo levantado por desenvolvedores é que devido ao fato da sintaxe da linguagem Go em alguns aspectos é um pouco extensa comparado com outras soluções em outras linguagens como scala e python, é necessário escrever um pouco mais de código, porém isso pode ser contornado fornecendo implementações específicas de operações recorrentes que possam ser reusadas evitando a necessidade de se implementar a mesma funcionalidade em diferentes contextos.

\section{Trabalhos Futuros}
\label{sec:future_work}

Ao longo do desenvolvimento da solução, alguns casos de uso interessantes foram detectados, propostos e discutidos como por exemplo a possibilidade de se implementar pipelines distribuídos, permitindo a execução de diferentes estágios do pipeline em um cluster de máquinas seguindo um modelo similar ao modelo MapReduce proposto por Google \cite{paper:google:map_reduce}. Apesar de ser um caso de uso muito interessante, a complexidade de implementação não justificou sua necessidade momentânea e portanto não foi considerado prioridade no desenvolvimento. Porém essa possibilidade não foi completamente descartada e fica proposta como trabalho futuro uma vez que existem necessidades reais que justificam o investimento em tal solução, especialmente nos domínios de Big Data aonde o volume de dados a serem processados é incrivelmente grande e a possibilidade de se distribuir e processar conjuntos menores destes dados em diferentes máquinas agregando seus resultados ao final é muito atraente.

%
% ---------------------------------------------------------
%

\bibliographystyle{abntex2-alf}
\bibliography{biblio}

\end{document}
